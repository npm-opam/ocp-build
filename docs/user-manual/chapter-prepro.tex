
\ocpbuild{} is part of \typerex{}. The simplest way to install it is
to use \opam{}\footnote{\url{http://opam.ocamlpro.com}}, the source package
manager for OCaml. If for some reasons, you are not satisfied by this
way, you will want to try to install it from its source repository, on
\github{}.

\section{Installing with \opam{}}

\ocpbuild{} is available in \opam{}. It is a meta-package (an empty
package) that triggers the installation of \typerex{}, with version
greater than 1.99. Indeed, \ocpbuild{} is compiled and installed by
the \typerex{} package. Any previous version of \ocpbuild{}
(especially version 0.1) should be uninstalled \underline{before}
installing \typerex{}.

First, let's check if ocp-build is already installed:
\begin{verbatim}
 peerocaml:~%  opam info ocp-build
             package: ocp-build
   installed-version: ocp-build.0.1 [4.00.1]
   available-version: 1.99.2-beta
         description: Project manager for OCaml
\end{verbatim}

The output of the command shows that \ocpbuild{} is already installed,
with version 0.1. We should remove it immediatly:

\begin{verbatim}
 peerocaml:~%  opam remove ocp-build
The following actions will be performed:
 - remove ocp-build.0.1
0 to install | 0 to reinstall | 0 to upgrade | 0 to downgrade | 1 to remove
\end{verbatim}

Note that some other packages depending on \ocpbuild{} can need to be
uninstalled too. You can keep a list of these packages, so that you
can install them again after installing the new version.

If you only ask \opam{} to install \ocpbuild{}, \opam{} might decide
to re-install \ocpbuild{} 0.1 because it has a shorter chain of
dependencies than \ocpbuild{} 1.99. To force it to install the new
version, we can ask for both \ocpbuild{} and \typerex{}:
\begin{verbatim}
 peerocaml:~%  opam install ocp-build typerex
The following actions will be performed:
 - install ocp-build.1.99.2-beta
 - install typerex.1.99.2-beta
2 to install | 0 to reinstall | 0 to upgrade | 0 to downgrade | 0 to remove
Do you want to continue ? [Y/n]
\end{verbatim}

\section{Installing from \github{}}

\ocpbuild{} sources can be retrieved from \github{}. The latest
version is developed in the {\tt typerex2} branch of the {\tt OCamlPro/typerex}
repository:

\begin{verbatim}
 peerocaml:~%  git clone git@github.com:OCamlPro/typerex.git
 peerocaml:~%  git checkout typerex2
\end{verbatim}

In the source directory ({\tt typerex}), We can now configure, compile
and install:
\begin{verbatim}
 peerocaml:~%  ./configure --prefix /usr/local/
 peerocaml:~%  make
 peerocaml:~%  make install
\end{verbatim}

The last command will install all \typerex{} commands and
libraries. If you just want to install \ocpbuild{}, you can use:

\begin{verbatim}
 peerocaml:~%  sudo ./_obuild/ocp-build/ocp-build.asm -install ocp-build \
    -install-bin /usr/local/bin -install-lib /usr/local/lib/ocaml
\end{verbatim}

Note that we used {\tt sudo} since the install paths we specified
require administrator priviledges.

It is also possible to uninstall files installed by {\tt make install}
using \ocpbuild{}:
\begin{verbatim}
 peerocaml:~%  ocp-build -uninstall typerex
\end{verbatim}

We can also use \ocpbuild{} to uninstall packages installed by
\ocpbuild{} (but it would be a bad idea to use that to uninstall
packages installed by \opam{}):
\begin{verbatim}
 peerocaml:~%  sudo ocp-build -uninstall ocp-build
\end{verbatim}


If you want to modify \ocpbuild{}, sources specific to \ocpbuild{} are
located in the {\tt tools/ocp-build} directory.
